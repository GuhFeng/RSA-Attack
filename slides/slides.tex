\documentclass{beamer}
\usepackage[utf8]{inputenc}
\usepackage{ctex}[UTF-8]
\usepackage[T1]{fontenc}
\usepackage{mathabx}
\usepackage{mathpazo}
\usepackage{eulervm}
\usepackage{natbib}


\usepackage[citations,footnotes,definitionLists,hashEnumerators,smartEllipses,tightLists=false,pipeTables,tableCaptions,hybrid]{markdown}


\markdownSetup{rendererPrototypes={
 link = {\href{#2}{#1}},
 headingOne = {\section{#1}},
 headingTwo = {\subsection{#1}},
 headingThree = {\begin{frame}\frametitle{#1}},
 headingFour = {\begin{block}{#1}}
}}


\usetheme{CambridgeUS}
\usefonttheme{serif}
\usecolortheme{default}


\title{RSA密码破译报告}
\author{冯古豪}
\institute{PKU EECS}

\begin{document}

\maketitle

\frame{\tableofcontents}

\begin{markdown}

# RSA加密原理
### 欧拉定理
\begin{theorem}
$a,\ n$为两个互素的正整数,则$a^{\phi(n)}\equiv 1\mod{n}$,其中$\phi(n)$为欧拉函数。
\end{theorem}
\end{frame}

### RSA加密
RSA加密算法首先生成出两个大素数$p,q$,要加密的二进制数记为$m$,公共模数$n=pq$,
公钥$e$为任意一个小于$\phi(n)$的正整数。私钥$d$满足$e\times d\equiv 1 \mod{\phi(n)}$,
加密后我们发送的数字$c=m^e \mod n$。在解密时,有$m=c^d\mod n$,其中,公钥为$c$公开的,
私钥$d$只有通信两方掌握。
\end{frame}

# 攻击方式和实验结果
## 基于低加密指数的攻击



### 低加密指数攻击

- 针对加密指数$e$很小的情况,一般为2,3
- 即使$e$很小,一般也不起作用

#### 算法原理
$m=\sqrt[e]{kn+c},\ k\in N$  
\newline
枚举$k$的取值来破解$RSA$

\end{block}
#### 实验结果
算法的时间复杂度过高,所以无法破译出任何一组数据。
\end{block}
\end{frame}


### 低加密指数广播攻击


- 中国剩余定理
- 相同的$m,e$,多个模数$n$

#### 算法原理
$m^e=\Sigma_{i=1}^{k} {c^{(i)}t^{(i)}N^{(i)}}+KN$
\newline
直接枚举$K$来完成破译


\end{block}
#### 实验结果
由于实验数据基本上满足$m\sim \Theta(n)$,所以枚举的$K$的范围接近$\mathcal{O}(n^{e-k})$,为了得到有效的时间复杂度,
该算法所需要数据的组数$k$需要满足$k\geqslant e $。
运用这个方法,我成功破译了$e=5$的五组数据。




\end{block}
\end{frame}



### Corper-Smith攻击
#### 算法原理

\end{block}
#### 实验结果

\end{block}
\end{frame}



## 维纳攻击
### 维纳攻击
#### 算法原理

\end{block}
#### 实验结果

\end{block}
\end{frame}



## 基于大数分解的攻击

### 费马分解
#### 算法原理

\end{block}
#### 实验结果

\end{block}
\end{frame}



### Pollard-$p-1$分解
#### 算法原理

\end{block}
#### 实验结果

\end{block}
\end{frame}


### Pollard-$\rho$分解
#### 算法原理

\end{block}
#### 实验结果

\end{block}
\end{frame}


## 其他攻击方式
### 公因数分解
#### 算法原理

\end{block}
#### 实验结果

\end{block}
\end{frame}


### 共模攻击
#### 算法原理

\end{block}
#### 实验结果

\end{block}
\end{frame}

\end{markdown}

\begin{frame}
    \begin{center}
        \huge{\textbf{谢谢大家}}
    \end{center}
\end{frame}

\end{document}
