\documentclass{beamer}
\usepackage[utf8]{inputenc}
\usepackage{ctex}[UTF-8]
\usepackage[T1]{fontenc}
\usepackage{mathabx}
\usepackage{mathpazo}
\usepackage{eulervm}
\usepackage{natbib}


\usepackage[citations,footnotes,definitionLists,hashEnumerators,smartEllipses,tightLists=false,pipeTables,tableCaptions,hybrid]{markdown}


\markdownSetup{rendererPrototypes={
 link = {\href{#2}{#1}},
 headingOne = {\section{#1}},
 headingTwo = {\subsection{#1}},
 headingThree = {\begin{frame}\frametitle{#1}},
 headingFour = {\begin{block}{#1}}
}}


\usetheme{CambridgeUS}
\usefonttheme{serif}
\usecolortheme{default}


\title{RSA密码破译报告}
\author{冯古豪}
\institute{PKU EECS}

\begin{document}

\maketitle

\frame{\tableofcontents}

\begin{markdown}

# RSA加密原理
### 欧拉定理
\begin{theorem}
$a,\ n$为两个互素的正整数,则$a^{\phi(n)}\equiv 1\mod{n}$,其中$\phi(n)$为欧拉函数。
\end{theorem}
\end{frame}

### RSA加密
RSA加密算法首先生成出两个大素数$p,q$,要加密的二进制数记为$m$,公共模数$n=pq$,
公钥$e$为任意一个小于$\phi(n)$的正整数。私钥$d$满足$e\times d\equiv 1 \mod{\phi(n)}$,
加密后我们发送的数字$c=m^e \mod n$。在解密时,有$m=c^d\mod n$,其中,公钥为$c$公开的,
私钥$d$只有通信两方掌握。
\end{frame}

# 攻击方式和实验结果
## 基于低加密指数的攻击



### 低加密指数攻击

- 针对加密指数$e$很小的情况,一般为2,3
- 即使$e$很小,一般也不起作用

#### 算法原理
$m=\sqrt[e]{kn+c},\ k\in N$  
\newline
枚举$k$的取值来破解$RSA$

\end{block}
#### 实验结果
算法的时间复杂度过高,所以无法破译出任何一组数据。
\end{block}
\end{frame}


### 低加密指数广播攻击


- 中国剩余定理
- 相同的$m,e$,多个模数$n$

#### 算法原理
$m^e=\Sigma_{i=1}^{k} {c^{(i)}t^{(i)}N^{(i)}}+KN$
\newline
直接枚举$K$来完成破译


\end{block}
#### 实验结果
由于实验数据基本上满足$m\sim \Theta(n)$,所以枚举的$K$的范围接近$\mathcal{O}(n^{e-k})$,为了得到有效的时间复杂度,
该算法所需要数据的组数$k$需要满足$k\geqslant e $。
运用这个方法,我成功破译了$e=5$的五组数据。




\end{block}
\end{frame}



### Corper-Smith攻击

- 针对$e=3$的情形,并且知道明文$m$的高位。


#### 算法原理
$$
m=M+x
$$
$$
(M+x)^3-c\equiv 0\mod n
$$
利用Copper-Smith算法,我们在较小的时间复杂度内解出满足条件$x^3-c\equiv 0\mod n$的最小的$x$。
\end{block}
#### 实验结果
利用Copper-Smith算法,破译了$e=3$的3组数据。
\end{block}
\end{frame}



## 维纳攻击
### 维纳攻击

#### 算法原理

$$
G=\gcd(p-1,q-1), \lambda(n)=\frac{\phi(n)}{G}
$$
$$
ed=\frac{K}{G}\phi(n)+1
$$
$$
\frac{e}{n}=\frac{k}{dg}\frac{\phi(n)}{n}+\frac{1}{dn}
$$
用$\frac{e}{n}$的连分数展开逼近$\frac{k}{dg}$求得$\phi(n)$
\end{block}
\end{frame}
### 维纳攻击

\begin{theorem}[Legendre]
    若满足$\vert\alpha-\frac{p}{q}\vert<\frac{1}{q^2},\ (p,q)=1$,则$\frac{p}{q}$是$\alpha$的连分数逼近。
\end{theorem}
\begin{theorem}[维纳攻击]
    当数据满足$q<p<2q$,$d<\frac{1}{3}n^{\frac{1}{4}}$的条件下,维纳攻击一定能够解出$p,q$。
\end{theorem}

#### 实验结果


在本次实验的数据中,数据均没有满足维纳攻击的条件,所以我使用维纳攻击没有破译出任何数据。
\end{block}
\end{frame}



## 基于大数分解的攻击

### 费马分解
#### 算法原理

费马分解是一种常见的大数分解的算法,由于$n=pq$,$p,q$均为素数。不妨假设$p\geqslant q$,
这时,我们有$n=(a-b)(a+b)$,$p=a+b,\ q=a+b$,从而有$a^2=n+b^2$。

\end{block}
#### 实验结果
\begin{itemize}
    \item $a=\sqrt[2]{n+b^2}$,我们可以枚举$b$,来完成费马分解。
    \item $b=\sqrt[2]{a^2-n}$,我们可以从$a=\lceil \sqrt{n}\rceil$枚举,每次加一,来完成费马分解。
\end{itemize}
我分别尝试了两种实现方式,发现前者在有效的时间内只能够破译1组数据,但是后一种实现方式能够破译3组数据。
\end{block}
\end{frame}



### Pollard-$p-1$分解
#### 算法原理

\begin{lemma}[Pollard-$p-1$]
    假设$n=pq$满足$(p-1)\vert B!$且$p,q$均为素数,则令$a=2^{B!}\mod n$,$p=\gcd(a-1,n)$。
\end{lemma}
Pollard-$p-1$算法先设置一个较大的$B$,然后计算出$B!,\ a$,然后计算$\gcd(a,n)$即可,若$\gcd(a,n)\neq 1$,则我们解出了$n$的
一个素因数$p$,而后用$q=n/p$即可分解$n$。若$\gcd(a,n)=1$,则算法无效。

\end{block}
#### 实验结果
在实验中,我将$B$的大小设置成了200000,而后针对每一组数据的$n$分别计算$a$,最终成功破译了3组数据。

\end{block}
\end{frame}


### Pollard-$\rho$分解
#### 算法原理
算法随机生成三个数$x,y,c$,而后,
然后我们不断更新迭代$y=y^2+c\mod n$,然后计算$x-y,n$是否有公因数,不断循环直到找到非平凡的公因数$p$,
然后利用$n=pq$,完成对$n$的分解。
\end{block}
#### 实验结果
在实验中,Pollard-$\rho$算法成功破译了三组数据。
\end{block}
\end{frame}


## 其他攻击方式
### 公因数分解
#### 算法原理
对于多组数据中的不同的$n$,尝试将它们两两之间求最大公因数,若存在两组$n$之间存在在非平凡的公因数,则这两组数据可以直接破译了。
\end{block}
#### 实验结果
存在两组数据的$n$之间有非平凡的公因数。
\end{block}
\end{frame}


### 共模攻击

\begin{theorem}[共模攻击]
    记$s,t$满足$se^{(1)}+te^{(2)}=1$,则$m\equiv c^{(1)s}c^{(2)t}\mod n$
\end{theorem}
    

#### 算法原理
首先利用欧几里得算法计算出$s,t$,计算出$c^{(1)s}c^{(2)t}\mod n$即可。
\end{block}

#### 实验结果
成功破译了两组数据。
\end{block}
\end{frame}

\end{markdown}

\begin{frame}
    \begin{center}
        \huge{\textbf{谢谢大家}}
    \end{center}
\end{frame}

\end{document}
